\section{Lessons learned}
\label{Conclusion_Lessons}
\subsection{Kravspecifikationen}
\label{Conclusion_Lessons_Krav}
I begyndelsen af projektet så vi kravspecifikationen som en række af fejlfri krav, som ikke kunne diskuteres eller bøjes. Vi forsøgte at følge kravspecifikationen i dets fulde omfang i stedet for at opfylde de krav, som vi mente var de vigtigste.

Da vi havde nærlæst kravspecifikationen, fandt vi ud af, at der var en del krav som var forkerte i forhold til det hvad IT-universitet skulle bruge og behov for. Dette førte til, at vi lavede nogle ændringer til data modellen\footnote{Ændringerne er beskrevet i kapitel \ref{Background}.}.

Lektien for os har været, at udvikleren skal være opmærksom på, hvilke krav kunden stiller, og være klar til at diskutere kravene med kunden, så man er sikker på, at det bedste produkt bliver lavet. Desuden bør man diskutere en prioriteringsliste med kunden, således at kunden er sikker på, at udvikleren laver det vigtigste først.

\subsection{Projekter af samme natur er ikke nødvendigvis ens}
\label{Conclusion_Lessons_Projekt}
Da vi begge havde været med til at udvikle et system til filmudlejning, endte vi med at undervurdere, hvor meget arbejde der lå i implementationen af vores design af booking-systemet.

Systemet var udviklet i C\# med en WCF-service hooket op til en Microsoft SQL-database med en tilhørende klient. Da dette system delte mange af de samme elementer som systemet, vi har udviklet i forbindelse med dette projekt, så vurderede vi, at det ikke ville være et stort problem at overføre meget af funktionaliteten til dette projekt.

Vi glemte at tage med i vores vurdering, at vi var fem personer på det forrige projekt, og derfor havde vi specialiseret os i forskellige dele af systemet. Da vi kun var to person til dette projekt betød det at vi manglede de personer fra det forrige projekt som havde specialiseret sig inde for de de som vi ikke haft fokus på. Dette betød, at vi selv blev nød til at tilegne os den viden, fra de felter som de havde stået for.

Hvilket gjorde at vores vurdering af hvor dyrt det var at lave den løsning vi havde valgt var helt forkert.

Lektien i denne forbindelse har været, at man skal passe på med at fejlvurdere, hvor meget erfaring fra tidligere projekter kan gavne en i et andet projekt, selvom projekterne er meget ens.

\subsection{En service er ikke altid den rigtige løsning}
\label{Conclusion_Lessons_Service}
I forbindelse med vores valg af implementationsstrategi, lavede vi en vurdering af, hvor meget det ville gavne kunden (ITU), hvis vi udviklede en service som en del af systemet. Vi vurderede, at det ville være en stor fordel for kunden, hvis det var nemt at udvide systemet.

Den første release, som vi udviklede i forbindelse med dette projekt, havde dog slet ikke brug for en service. Vi burde i stedet have koblet vores klient direkte til databasen. 
\\Arkitekturen, vi anvender i klienten, gør det relativt smertefrit at koble klienten op til en service i stedet for direkte til databasen, hvis systemet senere skulle udvides.

Vi tænkte ikke over, at projektperioden var begrænset, så vi vurderede kun omkostningerne i vores cost/benefit-analyse i forhold til hinanden. Dette betød, at vi ikke havde lavet en egentlig vurdering af, om det var muligt for os at nå udvikle både service og klient.

I fremtidige projekter vil vi sørge for at vurdere, om der overhovedet er brug for avancerede features i forhold til kundes behov.

\subsection{Projekt styring}
\label{Conclusion_Lessons_Styring}
Vores styring af projektet har været meget løs og parallel. 

Det havde været en fordel for vores projekt, hvis vi havde haft del milestones eller en iterativ styringsmodel. Vi ville have haft bedre styr på, hvor meget vi manglede på et givent tidspunkt i processen. Desuden ville vi haft mere materiale at få feedback på fra vores vejleder.

Da vi har arbejdet meget parallelt, har vi ikke været gode nok til at holde hinanden opdateret på, hvor man var i forbindelse med det, man arbejdede på. Når vi endelig tog beslutninger blev de tilgengæld ikke dokumenteret ordenligt.

Vi manglede et fælles beslutningspapir som var til råddighed for os begge, hvor vi skrev alle vores beslutninger ned i forhold til teknisk og grafisk design beslutninger. Dette ville også have hjulpet os i forhold til vores meget parallel arbejdsgang da man kun konsultere beslutningspapiret i stedet for at forstyre sin partner.

Lektien vi har lært i denne forbindelse har været, at selvom man arbejder parallelt er det stadig vigtigt at have fælles mål. Derudover er det vigtigt at sørge for, at der stadig bliver kommunikeret mellem gruppe medlemerne og sørge for at den kommunikation bliver dokumenteret så man kan støtte sig op af den senere i processen.