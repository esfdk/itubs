\chapter{Konklusion}
\label{Konklusion}
\section{Lessons learned}
\label{Konklusion_Lessons}
I løbet af projektforløbet var der ting som vi gerne ville have gjort anderledes og ting vi burde have overvejet før vi gav os i kast med dem. Vi vil i dette afsnit gennemgå de vigtigste ting vi har lært og hvordan det har påvirket projektet.

\subsection{Kravspecifikationen}
\label{Konklusion_Lessons_Krav}
Da vi startede projektet var vi alt for sikre på, at kravspecifikationen var en fejlfri vejledning som, hvis man fulgte ville sikre, at slut produktet opfyldte alle de krav som ITU har til et booking system.

Vi fandt ud af, at det ikke var helt rigtig da der var visse ting der skulle ændres i kravspecifikationen for, at der kunne implementeres et godt system. Vi endte derfor med at bruge tid på at designe ting fra kravspecifikationen som vi troede skulle bruges, men som vi blev nød til at skrotte da der skulle laves ændringer i kravspecifikationen.

Et andet problem vi havde med kravspecifikationen var, at vi så den som en vejledning til hvad der skulle inkluderes i første release af systemet og ikke som en vejledningen til hvad der skulle være i det færdige system.
Det gjorde, at vi prøvede at inkludere alle kravene i vores første release i stedet for kun at vælge de mest nødvendige, hvilket gjorde, at vi kom i tidsnød da vi prøvede at udvikle for mange funktionaliteter på for kort tid.

Det vi har lært er, at man skal være kritisk når man  læser kravspecifikationen således, at hvis der er ting der virker forkert kan de rettes tidligt i processen hvis det er en fejl. Derudover har vi også lært, at man skal dele kravspecifikationen op således, at man kan priotere hvad der er vigtigst at få implementeret først så man ikke prøver at udvikle et fuldt produkt i en iteration.

\subsection{Projekter af samme natur er ikke nødvendigvis ens}
\label{Konklusion_Lessons_Projekt}
Et stort problem som vi lærte meget af var forventningen til hvad det krævede at lave et bookingsystem. Vi havde tidligere begge været med til at udvikle et system til leje af film som også var skrevet i c\# med en bagved liggende service. Vi følte derfor vi havde en god basic for hvad sådan et projekt krævede og hvor meget tid der skulle bruges til at implementere det.

Det vi dog overså var at vi havde været fem mennesker til at udvikle systemet til booking af film, og der var derfor nogen af arbejdsområderne vi ikke havde den fornødende ekspertise til at implementere i vores eget system uden problemer.

Vi undervuderede også størrelsen af projektet i og med vi troede vi kunne genbrug en stor del af arkitekturen fra vores forrig projekt og på den måde gøre op for det mindre antal af gruppemedlemer, det viste sig dog, at selvom grundstruktur var ens for de to projekter var meget af funktionaliteten ikke ens og vi skulle derfor bruge mere tid på udviklingen end regnet med.

\subsection{En serviceløsning er ikke altid bedst}
\label{Konklusion_Lessons_Service}
Da vi satte os for at vælge hvilken implementationsstrategi vi skulle benytte os af var vi beggen enige om, at vi skulle have en bagvedligende service da det ville give brugeren mulighed for at arbejde videre med systemet, eksmepelvis hvis de gerne ville udvikle en mobil applikation.

Det vi ikke tænkte over var, at projekt periode var kort og vi derfor ikke ville have særlig lang tid til at udvikle funktionaliteter hvis der også skulle laves en service, vi endte derfor ud med en service og et system som ikke havde brug for en service for at fungerer.
Vi ville derfor hvis vi skulle lave et ligne projekt vælge at lave et system uden service som fokuserede mere på funktionalitet og proof of concept end at fokuser på godt teknisk design.

\subsection{Projekt styring}
\label{Konklusion_Lessons_Styring}
En ting som vi mener ville have gjort, at vi kunne have opnået mere i løbet af projekt perioden var projekt styring. Da vi begyndt at implementere systemet satte vi ingen mål for hvad der skulle gøres færdig til hvornår, vi forsøgte i stedet at lave det hele i en iteration.

Derudover havde vi også en seperation af arbejdsopgaver hvilket gjorde, at der var problemer som ikke blev opdaget før sent i processen, da vi ikke havde nogen fast styring blev mange små beslutninger taget hen over brodet og blev derfor sjældent dokumenteret.
\\ Hvis vi skulle lave projektet igen ville vi nok lave det over flere iterationer og sørge for, at alle beslutninger blev dokumenteret således at der ikke ville være forvirring om hvilke beslutninger der var blevet taget.   