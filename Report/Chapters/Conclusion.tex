\chapter{Konklusion}
\label{Konklusion}
\section{Lessons learned}
\label{Konklusion_Lessons}
I løbet af projektforløbet var der ting som vi gerne ville have gjort anderledes og ting vi burde have overvejet bedre før vi gav os i kast med dem. Vi vil i dette afsnit gennemgå de vigtigste ting vi har lært og hvordan det har påvirket projektet.

\subsection{Kravspecifikationen}
\label{Konklusion_Lessons_Krav}

Da vi startede projektet var vi alt for sikre på, at kravspecifikationen var en fejlfri vejledning som, hvis man fulgte ville sikre, at man endte ud med et færdigt produkt som opfyldte de krav som ITU har til et booking system. Vi fandt ud af, at det ikke var helt rigtig da der var visse ting der skulle ændres i kravspecifikationen for at kunne implementere et godt system, vi endte derfor med at bruge tid på design ting fra kravspecifikationen som vi troede skulle bruges men som vi blev nød til at skrotte da der skulle ændres i kravspecifikationen.

Et andet problem vi havde med kravspecifikationen var at vi så den som en vejledning til hvad der skulle inkluderes i første release og ikke at den i virkeligheden dækkede over alt hvad der skulle være i det endelig bookingsystem. Det faktum gjorde at vi prøvede at inkludere alle kravne i vores første release i stedet for kun at vælge de mest nødvendige, hvilket gjorde at vi kom i tidsnød da vi prøvede at udvikle for mange funktionaliteter på for kort tid.

Det vi har lært er at man skal være kritisk når man  læser kravspecifikationen således, at hvis der er ting der virker forkert kan man tidligt i processen få konfimeret om det er en fejl. Derudover har vi også lært at man skal dele kravspecifikationen op således, at man kan priotere hvad der er vigtigst at få implementeret først så man ikke prøver at udvikle et fuldt produkt på engang.

\subsection{Projekter af samme natur er ikke nødvendigvis ens}
\label{Konklusion_Lessons_Projekt}
Et stort problem som vi lærte meget af var forventningen til hvad det krævede at lave et bookingsystem. Vi havde tidligere begge været med til at udvikle et bookingsystem til film som også var skrevet i c\# som havde en bagved liggende service. Vi følte derfor vi havde en god basic for hvad sådan et projekt krævede og hvor meget tid der skulle bruges til at implementere det. Det vi dog overså var at vi havde været fem mennesker til at udvikle systemet til booking af film, og der var derfor nogen af arbejdsområderne vi ikke havde den fornødende ekspertise til at implementere i vores eget system uden problemer. Vi undervuderede også størrelsen af projektet i og med vi troede vi kunne genbrug en stor del af arkitekturen fra vores forrig projekt og på den måde gøre op for det mindre antal af gruppemedlemer, det viste sig dog, at selvom grundstruktur var ens for de to projekter var meget af funktionaliteten ikke ens og vi skulle derfor bruge mere tid på udviklingen end regnet med.

\subsection{En serviceløsning er ikke altid bedst}
\label{Konklusion_Lessons_Service}
Da vi satte os for at vælge hvilken implementationsstrategi vi skulle benytte os af var vi beggen enige om, at vi skulle have en bagvedligende service da det ville give brugeren mulighed for at arbejde videre med systemet, eksmepelvis hvis de gerne ville udvikle en mobil applikation. Det vi ikke tænkte over var at projekt periode var kort og vi derfor ikke ville have særlig lang tid til at udvikle funktionaliteter og derfor ikke ville nå at kunne drage størrest mulig nytte af en service.

hvis projekt perioden er kort, kunne det måske bedre have betalt sig ikke at udvikle en service da den tager tid at udvikle og i et så lille projekt ikke giver særlig stor benefit i forhold funktionalitet i systemet.

\subsection{Projekt styring}
\label{Konklusion_Lessons_Styring}

for lidt dokumentation over de små beslutninger vi tog igennem hele processen