\chapter{Usability Testing}
\label{Usability}
Dette kapitel beskriver, hvad vores strategi for usability testing har været samt resultaterne af vores tests.

\section{Test strategi}
\label{Usability_TS}
Kravspecifikationen beskriver, at det er nødvendigt at lave usability tests tidligt i processen for at bevise proof of concept. Vi lavede papermockups af vores skærmbilleder for at opfylde dette krav. 

Vi har valgt at lave to runder af usability tests.
\\Runde 1 bestod af tests med en papermockup på en enkelt bruger.
\\Runde 2 består af tests af det endelige produkt på flere brugere.

Vi valgte kun at teste én bruger i runde 1, da flere testpersoner tidligt i udviklingsfasen kan føre til en overvældende liste af problemer\cite[s. 416]{SL_UID}. Derudover er en enkelt testperson ofte nok til at afsløre de mest seriøse problemer i brugergrænsefladen.

Vi mener, at det havde været optimalt, hvis vi kunne have udført en usability test midtvejs i processen (mellem runde 1 og runde 2), men vi vurderede, at vi ikke ville få nok ud af testen, da det ville være svært at finde tid til at teste nok personer samt implementere de mulige ændringsforslag.

\subsection{Test cases}
\label{Usability_TS_TC}
Vi har to brugergrupper i vores usability tests. Der er almindelige brugere (studerende/undervisere) og administration (Facility Management/Receptionister). Til hver af disse brugergrupper har vi defineret syv test cases.

Test cases til almindelige brugere:
\begin{description}
\item[Test Case 1:] Du har booket et lokale fra 9-10 til et møde med din vejleder og du vil gerne have en diktafon med, så du kan optage mødet.
\item[Test Case 2:] Du skal holde et møde på tre timer om et par dage. Du skal i den sammenhæng booke et lokale til formålet.
\item[Test Case 3:] Et af dine gruppemedlemmer har meldt tilbage, at han skal aflevere sin datter i børnehaven. Mødet kan derfor først holdes fra klokken 10.
\item[Test Case 4:] Da mødet ligger om morgen, tænker du, at det vil være fornuftigt, hvis der var noget kaffe og morgenbrød klar.
\item[Test Case 5:] Du skal til dit projekt optage en kort reklamefilm og har derfor brug for et kamera.
\item[Test Case 6:] Et af dine gruppemedlemmer har kamera med, så du behøver ikke længere booke et hos ITU.
\item[Test Case 7:] Deltagerne til dit møde i 2a12 har desværre aflyst.
\end{description}
Test cases til administrationsbrugere:
\begin{description}
\item[Test Case 1:] En elev har kontaktet dig og gjort dig opmærksom på, at der ikke længere er en projektor i 2A12.
\item[Test Case 2:] ITU har til sin udstyrssamling erhvervet sig to nye kameraer.
\item[Test Case 3:] En student henvender sig til dig ved skranken og spørger om hans booking er blevet godkendt.
\item[Test Case 4:] Du har et møde  i de kommende dage og vil helst gerne holde det på 4. sal.
\item[Test Case 5:] Du skal afholde et fordrag for 30 mennesker den 24, tidspunktet er irelevant men der skal være plads til min. 30 mennesker.
\item[Test Case 6:] Lokalet 5a12 er ikke længere et privat lokale, men skal istedet registreres som et mødelokale.
\item[Test Case 7:] Lokalet 2a12 er ikke længere til rådighed i forbindelse med booking.
\end{description}

Vi har designet vores test cases således, at testpersonen ikke får at vide hvor i systemet, de skal udføre deres arbejdsopgave. Dette har vi gjort for at finde ud af, hvor intuitivt vores system er.
\\De 14 test dækker til sammen størstedelen af systemets workflows. Vi udfører testene som "think aloud" tests\cite[s. 421]{SL_UID}. "Think-aloud" tests består af at læse test casen op for testpersonen og bede dem tænke højt og forklare, hvad de gør under testen.

Efter testen får brugeren mulighed for at give input til, hvad de synes er godt og hvad der kan forbedres. Dette giver os mulighed for at få et indtryk af, hvad vi har overset i vores design og hvilke elementer, vi bør bruge flere steder i systemet.

I første runde af tests havde vi fokus på de store problemer i systemet, hvor vi i anden runde havde fokus på at følge op på de ændringer, vi havde lavet i forhold til resultaterne fra første runde. I anden runde var vi også interesserede i at finde ud af, om vores system er intuitivt og effektivt nok.

\section{Runde 1: Resultater og konsekvenser}
\label{Usability_R1}
Efter den første runde af tests var der fem problemer, problemerne er klassifikeret i forhold til skalaen fra \cite[s. 439]{SL_UID}:
\begin{description}
\item [Punkt 1:] Medium problem - Usikker på hvordan man vælger booking tidspunkter.
\item [Punkt 2:] Cumbersome - Brugeren synes det var besværligt at navigere imellem skærmbillederne.
\item [Punkt 3:] Minor problem - Brugeren fandt det ikke naturligt, at man skulle bruge lokalelisten til at ændre sin booking.
\item [Punkt 4:] Minor problem - Problem med at tilføje udstyr/forplejning.
\item [Punkt 5:] Positiv feedback på den generelle struktur af systemet.
\end{description}

\paragraph{Punkt 1}
Test personen havde problemer med at finde ud af, hvordan vores gitter til booking af lokaler fungerede. Personen troede ikke, man kunne klikke på felterne i gitteret, så der skulle lidt hjælp til, før personen forstod princippet.
\subparagraph{Konsekvenser og løsning}
Vi blev enige om, at dette problem kunne løses ved at indsætte en checkbox i felterne. Det vil gøre det mere intuitivt, når man skal trykke på tiderne.

\paragraph{Punkt 2}
Testpersonen følte, at der manglede navigeringsmuligheder mellem de forskellige skærmbilleder. Det var ikke intuitivt, at man skulle bruges tabsne til at skifte skærmbillede.
\subparagraph{Konsekvenser og løsning}
Vi løste problemet ved at lave en menubar, som var placeret nærmere midten af billedet, så den blev mere synlig.

\paragraph{Punkt 3}
Testpersonen fandt det ikke naturligt, at man skulle bruge lokalelisten til at ændre sin booking.
\subparagraph{Konsekvenser og løsning}
Da vi kun testede på én person og det er et minor problem, valgte vi ikke at gøre noget ved problemet. Hvis problemer opstår i yderligere usability tests, bør en ændring overvejes.

\paragraph{Punkt 4}
Testpersonen nævnte, at hun ikke mente, det burde være muligt at tilføje udstyr/forplejning til en booking før, den er blevet godkendt.
\subparagraph{Konsekvenser og løsning}
Vi valgte at imødekomme dette ved at implementere logik, så man ikke kan trykke på knapperne til at tilføje forplejning/udstyr, hvis man ikke har valgt en godkendt booking.

\paragraph{Punkt 5}
Testpersonen kunne lide den overordnede struktur af systemet og var positiv overfor booking funktionaliteten (gitteret). Personen skulle dog lære at bruge booking funktionaliteten.
\subparagraph{Konsekvenser og løsning}
Denne feedback havde stor indflydelse på, at vi valgte at beholde vores grundstruktur.

\section{Usability Test: Runde 2}
\label{Usability_R2}
Vi fik ikke mulighed for at udføre runde 2 af vores usability tests i tide. Da vi planlægger at arbejde videre på systemet efter denne rapport er afleveret, så vil vi følge op på vores strategi ved at lave runde 2 efter at de programmingstekniske fejl er rettet.

Som beskrevet i afsnit \ref{Usability_TS} består runde 2 af test med flere brugere. Der skal laves tests med 2-3 personer fra hver af de to testgrupper. Disse brugere skal udføre "think-aloud" tests som beskrevet i vores usability test strategi. Testcases fra runde 1 skal genanvendes. Desuden vil vi tage en feedback-session efter testen sammen med brugeren på samme måde, som gjort i forbindelse med runde 1.

Målet med runde 2 er, at problemerne fra runde 1 er blevet løst samt at der ikke er nogen major problem eller task failures.

\subsection{Mulige konsekvenser af usabilitytest runde 2}
\label{Usability_R2_cons}
Konsekvenserne af testresultaterne i runde 2 ville være forslag til ændringer til næste udgave af systemet. Hvis ingen af problemerne fra runde 1 gentager sig, går vi ud fra at det overordnede design/layout er i orden. 

I tilfælde af at der kun er cumbersome problemer i forbindelse med brugergrænsefladen, mener vi, at det vil være sandsynligt at systemet er være klar til brug i forhold til usability.