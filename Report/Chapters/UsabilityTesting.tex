\chapter{Usability testing}
I dette kapitel vil vi kigge på usability test af vores prototype og vores strategi for disse tests
\section{Test strategi}
Allerede meget tidligt i processen besluttede vi os for at det var vigtigt at få lavet usability test af vores system, både med superbruger og normalbruger i vores tilfælde receptionister og studerende. Vi startede derfor med at lave papermockups af vores skærmbilleder således at vi kunne lave usability test tidligt i forløbet så vi kunne nå at tage resultaterne med ind i udviklingsfasen. Til den første runde af tests valgt vi kun at teste på en bruger da det plejer at være mere end nok til at afsløre de seriøse problem i interfacet\footnote{User Interface Design, side 416}. Til test cases lavede vi 7 test cases til hver af de to brugergrupper.
Test cases til studerende:
\begin{itemize}
\item Test Case 1: Du har booket et lokale fra 9-10 til et møde med din vejleder og du vil gerne have en diktafon med så du kan optage mødet.
\item Test Case 2: Du skal holde et møde den 22 september klokken 9 og tre timer frem. Du skal i den sammenhæng booke et lokale til formålet.
\item Test Case 3: En af dine gruppemedlemer har meldt tilbage at han skal aflever sin datter i børnhavnen og mødet kan derfor først holdes fra klokken 10.
\item Test Case 4: Da mødet ligger om morgen tænker du at det vil være fornuftigt hvis der var noget kaffe og morgenbrød klar.
\item Test Case 5: Du skal til dit projekt optage en kort reklame film og har derfor brug for et kamera.
\item Test Case 6: Dit gruppe medlemer har kamera med så du behøver ikke længere booke et hos ITU.
\item Test Case 7: Deltagerne til dit møde i 2a12 har desværre aflyst.
\end{itemize}
Test cases til receptionisten:
\begin{itemize}
\item Test Case 1: En elev har kontaktet dig og gjort dig opmærksom på at der ikke længere er en projektor i 2A12
\item Test Case 2: ITU har til sin udstyrsamling erhvert sig to nye kameraer.
\item Test Case 3: En student henvender sig til dig ved skranken og spørger om hans booking er blevet godkendt.
\item Test Case 4: Du har et møde den 22 april og vil helst gerne holde det på 4. sal.
\item Test Case 5: Du skal afholde et fordrag for 30 mennesker den 24, tidspunktet er i relevant men der skal være plads til min. 30 mennesker.
\item Test Case 6: Lokalet 5a12 er ikke længere et privat lokale men er nu blevet lavet om til et mødelokale.
\item Test Case 7: Lokalet 2a12 er ikke længere til rådighed for bookning længere.
\end{itemize}
Test casene er designet sådan at de ikke leder test personen til at gøre noget specifikt, det er gjort fordi vi ikke ville have at test personen skulle få at vide hvad de skulle gøre for at løse problemet men mere selv finde løsning, da det ville give også et bedre indblik i om vores system nu var så intuitivt som vi følte det var. Derudover dækker de 14 test cases til sammen også størreste delen af workflowsne i systemet, hvilket gøre at vi for teste det hele når vi laver test på både en studerende og receptionist. Måden hvor på vi udførte vores tests var at vi læste test casen op for test personen og  bad dem så om tænke højt og forklare hvad de gjorde mens de prøvede at løse problemet. Når brugen så var færdig med test casene spurgte vi indtil hvad de synes der var godt og hvad de synes der kunne forbedres, hvilket gjorde at vi hele tiden havde en god ide om hvad der var værd at beholde og hvad der måske skulle kigges en ekstra gang på før det blev implementeret.

\section{Resultater af usabilitytest runde 1 }
Efter den første runde af usabilitytests var der 5 hoved punkter som vi synes var værd at tage med videre i udviklingsprocessen.
\begin{itemize}
\item Forvirring om hvordan man valgt tidspunkt at booke lokale.
Test personen havde problemer med at finde ud af hvordan vores gitter med booking tider fungerede, personen troede ikke man kunne klikke i selve felterne så der skulle lidt hjælp til før det de forstod princippet i det.
\item Navigering imellem de forskellige skærmbilleder
Test personen følte at der manglede navigierings muligheder mellem de forskellige skærmbilleder, opdagede ikke at der var tabs i toppen hvor man kunne skifte.
\item Synes ikke tilføj udstyr/forplejning var rigtig
Da vi spurgte receptionisten efter testen hvad hun synes der var dårligt nævnte hun, at hun ikke synes det skulle være muligt at tilføje udstyr og forplejning til sin bookning før receptionisten havde godkendt den. 
\item Forvirring om hvordan man sletter forplejning/Udstyr og booking
Da test personen fik til opgave at slette udstyr fra en booking, var der ingen der naturligt tænkte at de skulle gå ind i booking billedet og så klikke på de bookede tider som de gerne ville slette.
\item Kunne godt lide den generelle struktur af systemet
Begge test personen synes begge at vores gitter funktion var god efter de havde fået at vide at de kunne klikke på felterne, desuden synes de også det var godt at de forskellige skærmbilleder mindede om hinanden.
\end{itemize}

\section{Konsekvenser af usabilitytest runde 1}
Som nævnt ovenfor var der visse ting som test personer følte enten manglede eller ikke var intuitive nok

\section{Resultater af usabilitytest runde 2}
\section{Mulige konsekvenser af usabilitytest runde 2}