\chapter{Vurdering af vores løsning}
\label{Evaluation}
\section{Opfyldelse af kravspecifikation}
\label{Evaluation_KS}
Vi vurderer vores løsning ud fra en kravspecifikation, som er udarbejdet på kurset "Anskaffelse og kravspecifikation" (SANS) i efteråret 2012. Kravspecifikationen er baseret på Kravskabelon SL-07\footnote{http://www.itu.dk/people/slauesen/SorenReqs.html\#SL-07} (\copyright  Søren Lauesen, 2012).

\subsection{Nuværende og ønsket situation}
\label{Evaluation_KS_situation}


\subsection{Understøttede arbejdsopgaver}
\label{Evaluation_KS_workareas}
Kravspecifikationen definerer 13 arbejdsopgaver:
\begin{my_description}
\item[\textbf{C1.}]{Administrer booking}
\item[\textbf{C2.}]{Forespørg på booking}
\item[\textbf{C3.}]{Modtag forespørgsel fra Ekstern}
\item[\textbf{C4.}]{Klargør lokaler}
\item[\textbf{C5.}]{Udlevér nøgle}
\item[\textbf{C6.}]{Håndtér forplejning}
\item[\textbf{C7.}]{Fakturér forplejning}
\item[\textbf{C8.}]{Opdatér menukort}
\item[\textbf{C9.}]{Opdater listen for ekstra udstyr}
\item[\textbf{C10.}]{Opdatér lokaler}
\item[\textbf{C11.}]{Administrér bruger}
\item[\textbf{C12.}]{Håndter statistik}
\item[\textbf{C13.}]{Behandl faktura}
\end{my_description}

Ud af de 13 arbejdesopgaver understøtter vi \textbf{C1}, \textbf{C9} og \textbf{C10}.
\\Hver af arbejdsopgaverne har en række underpunkter, som beskriver delopgaver og varianter af disse delopgaver. 

\textbf{C1:} Administrer booking\\
\textbf{Brugere:} Ansatte og studerende på ITU.\\
\textbf{Hyppighed:} Flere gange om dagen i alt for både studerende og ansatte.\\
\textbf{Start:} En person går ind i booking systemet for at booke, ændre eller få overblik over bookinger.\\
\textbf{Slut:} Når bookingen eller lændringen er foretaget.

\begin{tabular}{ | p{7.5cm} | p{2.5cm} | p{2.2cm} | p{2.7cm} |}
\hline
\textbf{Arbejdesopgave} & \textbf{Eks. løsning} & \textbf{Understøttet}  & \textbf{Vores løsning}\\ 
\hline
1. Søg efter ledigt tidspunkt for ny booking & & Ja & Billede \ref{Design_G_Development_FinalGrid}(side \pageref{Design_G_Development_FinalGrid}) \\ 
\hline
1a. Søg efter eksisterende booking &  & Ja & Billede \ref{Design_G_Development_YourBookings_Final}(side \pageref{Design_G_Development_YourBookings_Final}) \\ 
\hline
2. Vælg ledigt lokale, forplejning og ekstra udstyr & & Ja & Billede \ref{Design_G_Development_Forplejning_Final}(side \pageref{Design_G_Development_Forplejning_Final})\\ 
\hline
2a. Vis eksisterende booking(er) & & Ja & Billede \ref{Design_G_Development_FinalGrid}(side \pageref{Design_G_Development_FinalGrid})\\ 
\hline
3. Opdater booking & & Ja & Billede \ref{Design_G_Development_FinalGrid}(side \pageref{Design_G_Development_FinalGrid})\\ 
\hline
4. Annuller booking & & Ja & Billede \ref{Design_G_Development_YourBookings_Final}(side \pageref{Design_G_Development_YourBookings_Final})\\ 
\hline
5. Opret booking & & Ja & Billede \ref{Design_G_Development_FinalGrid}(side \pageref{Design_G_Development_FinalGrid})\\ 
\hline
\end{tabular}

\textbf{C9:} Opdater listen for ekstra udstyr\\
\textbf{Brugere:} Facility Management (FM)\\
\textbf{Hyppighed:} Når der sker ændring i udvalget af ekstra udstyr (en gang om måneden)\\
\textbf{Start:} Når der er behov for at udstyrslisten skal opdateres\\
\textbf{Slut:} Når listen er opdateret

\begin{tabular}{ | p{7.5cm} | p{2.5cm} | p{2.2cm} | p{3.5cm} |}
\hline
\textbf{Arbejdesopgave} & \textbf{Eks. løsning} & \textbf{Understøttet}  & \textbf{Vores løsning}\\ 
\hline
C9.1 Rediger udstyrslisten & FM er ansvarlig & Ja & Billede \ref{Design_G_Development_UdstyrsListe_Final}(side \pageref{Design_G_Development_UdstyrsListe_Final}) \\ 
\hline
\end{tabular}

\textbf{C10:} Opdatér lokaler\\
\textbf{Brugere:} Facility Management (FM)\\
\textbf{Hyppighed:} Når der sker ændring i antallet af lokaler til bookingen (en gang om måneden)\\
\textbf{Start:} Når der er behov for at lokalelisten skal opdateres\\
\textbf{Slut:} Når listen er opdateret

\begin{tabular}{ | p{7.5cm} | p{2.5cm} | p{2.2cm} | p{3.2cm} |}
\hline
\textbf{Arbejdesopgave} & \textbf{Eks. løsning} & \textbf{Understøttet}  & \textbf{Vores løsning}\\ 
\hline
C10.1 Rediger lokalelisten & & Ja & Billede \ref{App_GUI_final_LokaleListe}(side \pageref{App_GUI_final_LokaleListe}) \\ 
\hline
C10.2 Rediger lokale inventar & & Ja & Billede \ref{Design_G_Development_AendreLokale_Final}(side \pageref{Design_G_Development_AendreLokale_Final}) \\ 
\hline
\end{tabular}

\subsubsection{Prioritering af arbejdsopgaver}
\label{Evaluation_workareas_priorities}
Vi valgte til den første release af systemet at priotere

Til den første release af systemet valgte vi at fokusere på at understøtte systemadministration og booking af lokaler og udstyr, hvilket tydeligt kan ses i afsnit \ref{Baggrund_Arb_opgaver} da alle arbejdsopgaverne i arbejdsområderne \textbf{C1},\textbf{C9} og \textbf{C10} er understøttet. 

I kravspecifikationen er de arbejdsområder også blandt dem som er vægtet højest og det er samtidig de arbejdsområder som indeholder kernefunktionerne i systemet. Det som ikke blev prioriteret højt nok til at komme med i første release var de arbejdsopgaver som fokuserede på integrationen med kantinen, samt mulighed for fakturing og statestik , arbejdsopgaverne \textbf{C6}, \textbf{C7}, \textbf{C12} og \textbf{C13} er eksempler på sådanne opgaver.
\\Grunden til, at vi ikke tog de arbejdsopgaver med var, at vi vurderede, at de ikke var nødvendige for at have et system der understøttede basis funktioner, derudover lagde arbejdsopgaverne op til, at der skulle implementeres et interface specifikt til kantinen hvilket vi vurderede ville tage lang tid at implementere og vi ville også have tre brugergrupper at tage hensyn til ift. usability og testing.

Fokus for næste release vil derfor være at få udarbejdet et interface som kantinen kan bruge til at integrere med resten af systemet og få finpudset de allerede eksisterende funktioner i kravspecifikationen. Behandling af faktura har også en høj vægtning, så det vil der også blive lagt fokus på. 


%FUTURE RELEASE NOTE: Der skal være flere rettighedsproperties i AD!!!!
\section{Fremtidig design}
\label{Design_G_Future}
Dette afsnit er en liste af vores forslag til ændringer i brugergrænsefladen. Det er ikke en prioteret liste, da alle ændringerne er quality of life forbedringer.

\begin{my_itemize}
\item{Brugerens egne bookinger skal være øverst på listen over bookinger}
\item{Sammenlægning af "Booking-Liste" og "Find Bookinger"}
\item{Sammenlægning af "Konfigurering af Lokaler" og "Ændre Lokale"}
\item{Informering om ændringer i status på en brugers booking}
\item{Understøt godkendelse af forplejning og udstyrsvalg}
\item{Mulighed for at skifte mellem sprog (dansk/engelsk)}
\item{Tilføjelse af nye typer forplejning og nye udstyrstyper}
\item{Mulighed for ændre priser på forplejning}
\item{Være i stand til at søge på sal i lokalelisten}
\item{Mulighed for at angive antal deltagere til en booking}
\item{Mulighed for at administrator kan ændre bookinger}
\item{Sørg for at der paging på alle sider med gittrer}
\end{my_itemize}
