\section{Kodeeksempler}
\label{Technical_CodeExamples}
Vi har udvalgt tre dele af koden, som viser vigtige dele af vores implementation.
\subsection{Active Directory tilgang}
\label{Technical_CodeExamples_AD}
Vi tilgår ITUs active directory i forbindelse med login og oprettelse af nye brugere i systemet. Kodestykke \ref{AD-AUTH} og \ref{AD-GETINFO} køres i samme login metode, men vi har delt dem op for læsbarhedens skyld.
\subsubsection{Godkendelse af login via AD}
\label{Technical_CodeExamples_AD_authen}
\begin{lstlisting}[caption=Kode til godkendelse af login, label=AD-AUTH]

var ldap = new LdapConnection(Configuration.LDAPServer) { Credential =  new NetworkCredential(username.Substring(0, username.IndexOf("@")), password) };
try{
    ldap.Bind();
}
catch (COMException){
    return new Person { ID = -1 };
}
catch (Exception e){
    return e.Message.Contains("available") ? new Person { ID = -1 } : null;
}
\end{lstlisting}
Vi godkender login oplysninger igennem koden i kodestykke \ref{AD-AUTH} (side \pageref{AD-AUTH}). 
\\Hvis det ikke er muligt at lave et \textit{Bind()} til LDAP-serveren, returnerer vi -1 (anvendes andetsteds i koden). Hvis brugernavn og/eller password er forkert, returnerer vi og stopper login proceduren.

Hvis det er muligt at kontakte LDAP-serveren og brugernavn/password godkendes, går vi videre til koden i kodestykke \ref{AD-GETINFO} for at finde den information om brugeren, der er brug for.

\subsubsection{Hentning af brugerdata gennem AD}
\label{Technical_CodeExamples_AD_userdata}
\begin{lstlisting}[caption=Kode til at hente bruger information, label=AD-GETINFO]

var dsFilter = "(mail=" + username + ")";
var de = new DirectoryEntry("LDAP://" + Configuration.LDAPServer, username.Substring(0, username.IndexOf("@")), password);

var ds = new DirectorySearcher(de){ Filter = dsFilter};

SearchResult result = null;
try{
   result = ds.FindOne();
}
catch (COMException){
   return new Person { ID = -1 };
}

if (result == null){
   return null;
}

if (!result.Properties.Contains("ituAffiliation") || !result.Properties.Contains("mail")){
   return new Person {
              Email = string.Empty, 
	   Name = string.Empty, 
	   Token = string.Empty 
	};
}

if (All.Any(p => p.Email.Equals(result.Properties["mail"][0].ToString()))){
   var person = All.First(p => p.Email.Equals(result.Properties["mail"][0].ToString()));
   person.Token = GenerateToken();
   BookITContext.Db.SaveChanges();
   return person;
}
else{
   var person = new Person{Email = result.Properties["mail"][0].ToString(), Name = result.Properties["displayName"][0].ToString()};
    person.Roles.Add(new Role{ RoleName = result.Properties["ituAffiliation"][0].ToString()});
   person.Token = GenerateToken();
   BookITContext.Db.People.Add(person);
   BookITContext.Db.SaveChanges();
   return All.FirstOrDefault(p => p.Email.Equals(person.Email));
}
\end{lstlisting}


\subsection{Trådsikker tilgang til database}
\label{Technical_CodeExamples_threadstatic}
Vores database kald er trådsikre, da kald til databasen fra \textit{Entity Framework} sørger for, at ændringer i databasen er atomic (i forbindelse med et \textit{SaveChanges} metode kald i \textit{Entity Framework}.). 
Da vores objektkontekst er "ThreadStatic" (se  linje 3 i kodestykke \ref{Code-ThreadStatic}, side \pageref{Code-ThreadStatic}). "ThreadStatic" elementer bliver først instatieret, når en ny tråd bliver oprettet. Derfor bruger vi en get-property til at instantiere vores objektkontekst. 
\\En problematik med dette er, at to tråde kan have forskelligt indhold i deres kontekster, og derfor ikke viser helt opdateret data til brugeren.En løsning på dette kunne være, at man opdaterede objektkonteksten, hver gang man foretog ændringer til databasen eller efter bestemte tidsintervaller.
\begin{lstlisting}[caption=Trådsikkerhed i tilgang til databasen, label=Code-ThreadStatic]

[ThreadStatic]
private static BookITContext db;

public static BookITContext Db
{
    get{
         return db ?? (db = new BookITContext());
    }
}
\end{lstlisting}

\section{Git repository og database/service information}
\begin{my_description}
\item[GitHub repository] https://github.com/esfdk/itubs
\item[Service URL] http://rentit.itu.dk/RentIt12/Services/Service.svc
\item[Database] RentIt12 på http://rentit.itu.dk (Brugernavn: RentIt12Db -- Adgangskode: Zaq12wsx)
\item[Test Administrator] Brugernavn="Admin@BookIt.dk" Adgangskode="zAq12wSx"
\item[Test Bruger] Brugernavn="test@bookit.dk" Adgangskode="qwerty123"
\end{my_description}