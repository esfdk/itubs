\chapter{Design Af Brugergrænsefladen}
\label{Design_G}
Dette kapitel beskriver det generelle design af brugergrænsefladen samt de beslutninger, som ligger bag. Alle vores papermockups og endelige skærmbilleder kan findes i appendix, afsnit \ref{App_GUI} på side \pageref{App_GUI}.

\section{Generelle mål}
\label{Design_G_Goals}
Vi har valgt at designe vores brugergrænseflade ud fra reglerne om design af virtuelle vinduer\cite[s. 169]{SL_UID} samt Ease Of Use principperne\cite[s. 9]{SL_UID}. I forbindelse med dette valg har vi sat følgende mål for designet:
\begin{itemize}
\item Strømlinet brugergrænseflade
\item Få forskellige skærmbilleder
\item Overblik
\item Effektivt
\end{itemize}

\subsection{Strømlinet brugergrænseflade}
Vi har valgt at designe skærmbillederne med samme grundstruktur. Denne lighed bør gøre det intuitivt at gå fra et skærmbillede til et andet i forbindelse med udførsel af opgaver. Desuden følger det reglen om få vindueskabeloner.

\subsection{Kort vej fra en opgave til en anden}
Brugergrænsefladen skal gøre det hurtigt og nemt for brugeren at komme fra en opgave til en anden. Dette skal gøres ved at have få skærmbilleder involvereret i en enkelt task (reglen om få vinduer per opgave).

\subsection{Overblik}
Brugeren skal have mulighed for nemt at danne sig overblik over bookinger, udstyr og forplejning\footnote{Reglen om den nødvendige oversigt af data.}. Derfor skal vi have seperate skærmbilleder, som giver overblik over hver type.

\subsection{Effektivt}
Det skal være effektivt at udføre opgaver for brugere, som anvender systemet ofte.

\section{Brugergrænsefladens udvikling og udseende}
\label{Design_G_Development}
Vores skærmbilleder er opdelt i tre typer: Gitter, Almindelig og Pop-ups. 
\\Gitterskærmbillederne bruger vi til booking af lokale, forplejning og udstyr samt administration af bookinger, lokaleinventar og udstyr.
\\Almindelige vinduer anvender vi i forbindelse med login eller, hvis man skal ændre noget på et stykke udstyr/inventar.
\\Pop-up skærmbilleder er generelt advarsler eller fejlbeskeder.

Vores første mockup af skærmbilledet til booking af lokaler (figur \ref{Design_G_Development_FirstGrid}) havde et gitter, hvor hver række var et lokale og tiderne var kolonner. Man skulle klikke i et felt for at vise, at man ønskede at booke på et bestemt tidspunkt. Når man havde valgt de tider og lokaler, man gerne ville booke, skulle man trykke på en "Book" knap. 

Vi valgte at beholde gitterstrukturen. Vi tilføjede dog checkboxe til gitteret for at imødekomme et problem, som vi observerede i forbindelse med vores usability test\footnote{Se afsnit \ref{Usability_R1} på side \pageref{Usability_R1}}. Figur \ref{Design_G_Development_FinalGrid} viser vores endelige version.

\begin{figure}[h!]
  \centering
    \includegraphics[angle=90, width=0.65\textwidth]{Appendix/GUI-Prototype/PaperMockup/LokaleListe_001}
  \caption{Første udgave af gitter layoutet}
\label{Design_G_Development_FirstGrid}
\end{figure}

\begin{figure}[h!]
  \centering
    \includegraphics[width=0.6\textwidth]{Appendix/GUI-Prototype/DigitalMockup/GridEksempel}
  \caption{Skærmbilledet til booking af lokaler}
\label{Design_G_Development_FinalGrid}
\end{figure}

Når man har valgt lokale og tidspunkt, bliver man sendt videre til en oversigt over ens egne bookinger (se figur \ref{Design_G_Development_YourBookings_Final}).

\begin{figure}[h!]
  \centering
    \includegraphics[width=0.6\textwidth]{Appendix/GUI-Prototype/DigitalMockup/DineBookinger}
  \caption{Skærmbilledet til visning af bookinger}
\label{Design_G_Development_YourBookings_Final}
\end{figure} 

Figur \ref{Design_G_Development_Forplejning_Final} viser det endelige skærmbillede til valg af forplejning. 
Vi fokuserede på, at skærmbilledet skulle minde om \ref{Design_G_Development_FinalGrid}, så vi holdte os til reglen om få vindueskabeloner. De samme designbeslutninger gælder for skærmbilleder til tilføjelse af udstyr til en booking (se appendix).

\begin{figure}[h!]
  \centering
    \includegraphics[width=0.6\textwidth]{Appendix/GUI-Prototype/DigitalMockup/Forplejning}
  \caption{Skærmbilledet til booking af forplejning}
\label{Design_G_Development_Forplejning_Final}
\end{figure} 

\subsection{Bookingliste}
Figur \ref{Design_G_Development_BookingListe} viser vores papermockup til godkendelse af bookinger. 
I papermockup havde vi godkend/afvis-knapperne inde i selve gitteret. Dette besluttede vi dog kunne blive forvirrende, så den endelige version af skærmbilledet (figur \ref{Design_G_Development_BookingListe_Final}) har knapperne ved siden af gitteret. Da ingen af de andre skærmbilleder har egentlige knapper inde i gitteret, opfylder i større grad reglen om få vindueskabeloner.

\begin{figure}[h!]
  \centering
    \includegraphics[width=0.6\textwidth]{Appendix/GUI-Prototype/DigitalMockup/BookingListe}
  \caption{Skærmbilledet af recepsionistens liste af bruger bookinger}
\label{Design_G_Development_BookingListe_Final}
\end{figure} 

\begin{figure}[h!]
  \centering
    \includegraphics[width=0.7\textwidth]{Appendix/GUI-Prototype/PaperMockup/GodkendBookinger_001}
  \caption{Papermockup af recepsionistens liste af bruger bookinger}
\label{Design_G_Development_BookingListe}
\end{figure} 

\subsection{Udstyrsliste}
Figur \ref{Design_G_Development_UdstyrsListe_Final} viser det endelige skærmbillede af listen over udstyr.
Skærmbilledet indeholder både overblikket over alt det udstyr, som er tilrådighed, og muligheden for at registrere nyt udstyr til systemet.
\\Vi delte de to funktionaliteter op, så det var tydeligt, at det er seperate funktionaliteter.

Den eneste ændring siden vores papermockup (se appendix) er tilføjelsen af en checkbox, hvor man kan vælge et nyt stykke udstyr skal være til udlån. Dette betyder, at vi sparer et skærmbillede væk. Da udstyr og inventar er stort set ens i systemet, kan vi afgøre, om det nye element er inventar eller udstyr gennem denne checkbox.

\begin{figure}[h!]
  \centering
    \includegraphics[width=0.6\textwidth]{Appendix/GUI-Prototype/DigitalMockup/UdstyrsListe}
  \caption{Skærmbilledet af recepsionistens liste over udstyr i ITUs system.}
\label{Design_G_Development_UdstyrsListe_Final}
\end{figure} 

\subsection{Ændring af lokale}
Skærmbilledet til ændring af lokale er designet således, at man kan ændre navn og kapacitet på lokalet samtidig med, at man kan få et overblik over udstyret, som er tilføjet til lokalet.

Papermockup af dette skærmbillede (figur \ref{Design_G_Development_AendreLokale}) var delt op i tre dele: et til at skifte navn/kapacitet, et til at tilføje inventar og et til at fjerne inventar.
\\Dette var meget uoverskueligt og mindede ikke om resten af vores design. Vi valgte derfor at bruge gitterløsningen til udstyret. Den øverste del af gitteret viser det inventar, som er tilføjet lokalet. Resten af gitteret viser det inventar, som ikke er tildelt noget lokale. Teksten på "Tilføj Udstyr"-knappen ændrer sig til "Fjern Udstyr", hvis man trykker på en linje i den øverste del af gitteret.

\begin{figure}[h!]
  \centering
    \includegraphics[width=0.6\textwidth]{Appendix/GUI-Prototype/DigitalMockup/AendreLokale}
  \caption{Skærmbilledet til ændring af lokale.}
\label{Design_G_Development_AendreLokale_Final}
\end{figure} 

\begin{figure}[h!]
  \centering
    \includegraphics[angle=90, width=0.5\textwidth]{Appendix/GUI-Prototype/PaperMockup/AendreLokale_001}
  \caption{Papermockup til ændring af lokale.}
\label{Design_G_Development_AendreLokale}
\end{figure} 

\section{Fremtidig design}
\label{Design_G_Future}
Dette afsnit er en liste af vores forslag til ændringer i brugergrænsefladen. Det er ikke en prioteret liste, da alle ændringerne er quality of life forbedringer.

\begin{my_itemize}
\item{Brugerens egne bookinger skal være øverst på listen over bookinger}
\item{Sammenlægning af "Booking-Liste" og "Find Bookinger"}
\item{Sammenlægning af "Konfigurering af Lokaler" og "Ændre Lokale"}
\item{Informering om ændringer i status på en brugers booking}
\item{Understøt godkendelse af forplejning og udstyrsvalg}
\item{Mulighed for at skifte mellem sprog (dansk/engelsk)}
\item{Tilføjelse af nye typer forplejning og nye udstyrstyper}
\item{Mulighed for ændre priser på forplejning}
\item{Være i stand til at søge på sal i lokalelisten}
\item{Mulighed for at angive antal deltagere til en booking}
\item{Mulighed for at administrator kan ændre bookinger}
\end{my_itemize}