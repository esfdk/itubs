\chapter{Design af brugergrænsefladen}
\label{Design_G}
Dette kapitel beskriver det generelle design af brugergrænsefladen samt de beslutninger, som ligger bag.

\section{Generelle mål}
\label{Design_G_Goals}
Vi har valgt at designe vores brugergrænseflade ud fra reglerne om design af virtuelle vinduer\cite[s. 169]{SL_UID} samt Ease Of Use principperne\cite[s. 9]{SL_UID}. I forbindelse med dette valg har vi sat følgende mål for designet:
\begin{itemize}
\item Strømlinet brugergrænseflade
\item Få forskellige skærmbilleder
\item Overblik
\item Effektivt
\end{itemize}

\subsection{Strømlinet brugergrænseflade}
Vi har valgt at designe skærmbillederne med samme grundstruktur. Denne lighed bør gøre det intuitivt at gå fra et skærmbillede til et andet i forbindelse med udførsel af opgaver. Desuden følger det designregel 1\footnote{Few window templates} om få vindueskabeloner.

\subsection{Kort vej fra en opgave til en anden}
Brugergrænsefladen skal gøre det hurtigt og nemt for brugeren at komme fra en opgave til en anden. Dette skal gøres ved at have få skærmbilleder involvereret i en enkelt task (designregel 2\footnote{Few window instances per task.}).

\subsection{Overblik}
Brugeren skal have mulighed for nemt at danne sig overblik over bookinger, udstyr og forplejning (regel 6\footnote{Necessary overview of data}). Derfor skal vi have seperate skærmbilleder, som giver overblik over hver type.

\subsection{Effektivt}
Det skal være effektivt at udføre opgaver for brugere, som anvender systemet ofte.

\section{Brugergrænsefladens udvikling og udseende}
\label{Design_G_Development}
Vores skærmbilleder er opdelt i tre typer: Gitter, Almindelig og Pop-ups. 
\\Gitterskærmbillederne bruger vi til booking af lokale, forplejning og udstyr samt administration af udstyr og lokaleinventar.
\\Almindelige vinduer anvender vi, hvis man skal ændre noget på et stykke udstyr/inventar eller et lokale.
\\Pop-up skærmbilleder er generelt advarsler eller fejlbeskeder.

\subsection{Gitterskærmbilledet}
\label{Design_G_Development_Grid}
Gitterskærmbillederne er de skærmbilleder som står for at visse hvad der kan bookes i vores system, lokaler, udstyr og forplejning.
Figur \ref{Design_G_Development_FinalGrid}, side \pageref{Design_G_Development_FinalGrid} og Figur\ref{Design_G_Development_FirstGrid}, side \pageref{Design_G_Development_FirstGrid} er screenshots af vores første udkaste til gitterbilledet og vores endelige design.
\begin{figure}[h!]
  \centering
    \includegraphics[width=0.5\textwidth]{Appendix/GUI-Prototype/DigitalMockup/GridEksempel}
  \caption{Den endelige udgave af gitteret}
\label{Design_G_Development_FinalGrid}
\end{figure}

\begin{figure}[h!]
  \centering
    \includegraphics[width=0.5\textwidth, angle=90]{Appendix/GUI-Prototype/PaperMockup/LokaleListe_001}
  \caption{Første udgave af gitter layoutet}
\label{Design_G_Development_FirstGrid}
\end{figure}

Vores første mockup af gitterskærmbilledet havde et gitter, hvor hver række var et lokale og tiderne var kolonner. Brugerne skulle klikke i et felt for at vise, at man ønskede at booke på et bestemt tidspunkt. Når man havde valgt de tider og lokaler, man gerne ville booke, skulle man trykke på en "Book" knap. Som man kan se på Figur \ref{Design_G_Development_FinalGrid} valgte vi at behold den struktur, men vi tilføjede checkboxses til gitteret. Dette blev gjort efter første rund af usability tests hvor vi observerede, at brugeren var i tvivl om hvor man skulle klikke for at vælge tider for en booking, derudover tilføjede vi også, at udstyr m.m., man har booket, ligger øverst i gitteret, når man får oversigten over fx udstyr. Dette gør det nemt at finde det element, man har booket/bestilt.


\subsection{Almindelige skærmbilleder}
\label{Design_G_Development_NormalWindows}
Disse typer af skærmbilleder er primært til administration af udstyr/inventar/lokale og lignende. Et eksempel er skærmbilledet til ændring af et stykke udstyr. 

\begin{figure}[h!]
  \centering
    \includegraphics[width=0.5\textwidth]{Appendix/GUI-Prototype/PaperMockup/AendreUdstyr_001}
  \caption{Papermockup af skærmbilledet til ændring af udstyr}
\label{Design_G_Development_EquipmentChange}
\end{figure} 

\begin{figure}[h!]
  \centering
    \includegraphics[width=0.5\textwidth]{Appendix/GUI-Prototype/DigitalMockup/AendreUdstyr}
  \caption{Skærmbilledet til ændring af udstyr}
\label{Design_G_Development_EquipmentChange_Final}
\end{figure} 

\subsubsection{Ændre udstyr}
Som man kan se på Figur \ref{Design_G_Development_EquipmentChange} var det første skærmbillede til Ændreudstyr smalt i forhold til det endelig, Figur \ref{Design_G_Development_EquipmentChange_Final}. Gruden til det blev ændret skyldtes at vi fuldte designregel1 og derfor gerne ville have at billederne var visuelt konsistense. 

\begin{figure}[h!]
  \centering
    \includegraphics[width=0.5\textwidth]{Appendix/GUI-Prototype/PaperMockup/DineBookinger_001}
  \caption{Papermockup af skærmbilledet til visning af bookinger}
\label{Design_G_Development_YourBookings}
\end{figure} 

\begin{figure}[h!]
  \centering
    \includegraphics[width=0.5\textwidth]{Appendix/GUI-Prototype/DigitalMockup/DineBookinger}
  \caption{Skærmbilledet til visning af bookinger}
\label{Design_G_Development_YourBookings_Final}
\end{figure} 

\subsubsection{Dine bookinger}
Som man kan se på Figurerne \ref{Design_G_Development_YourBookings} og \ref{Design_G_Development_YourBookings_Final} var den eneste ændring ti den endelig version af "Dine bookinger" at vi fjernede forklaringen til hvad de forskellige statuser repræsenterede. Vi har i designet af skræmbilledet\ref{Design_G_Development_YourBookings_Final} fokuset på at overholde Gestalt-lovene\cite[s. 68]{SL_UID}, specifikt Law of Proximity\footnote{"Pieces that are close together are perceived as belonging together".}, så det virkede naturligt at knapperne til højre hørte til gitteret. 

\begin{figure}[h!]
  \centering
    \includegraphics[width=0.5\textwidth]{Appendix/GUI-Prototype/PaperMockup/GodkendBookinger_001}
  \caption{Papermockup af skærmbilledet til visning af bruger bookinger}
\label{Design_G_Development_ApproveBookings}
\end{figure} 

\begin{figure}[h!]
  \centering
    \includegraphics[width=0.5\textwidth]{Appendix/GUI-Prototype/PaperMockup/GodkendBookinger_001}
  \caption{Skærmbilledet til visning af bruger bookinger}
\label{Design_G_Development_ApproveBookings_Final}
\end{figure}

\subsubsection{Godkend bookinger}
Figuerne \ref{Design_G_Development_ApproveBookings} og \ref{Design_G_Development_ApproveBookings_Final} viser henholdsvis papermockupen og det færdige skærmbilledede af "Godkend booking" skærmbillede der viser listen over hvilke bookinger der mangler at blive godkendt eller afvist af receptionisten. Forskellen fra papermockupen til det færdige skærmbillede er, at vi har valgt at flytte godkend og afvis knapperne fra at være tilknyttet til hver enkelt booking til at være uden for gitteret, således at vi ligsom Figur\ref{Design_G_Development_YourBookings_Final} benytter "Law of proximity", desuden havde ingen andre gittre knapper i sig, så vi vudere at det burde være ens på alle billeder således at vi overholdt designregel1.

\begin{figure}[h!]
  \centering
    \includegraphics[width=0.5\textwidth]{Appendix/GUI-Prototype/PaperMockup/UdstyrsListe}
  \caption{Papermockup af skærmbilledet til tilføjelse og visning af udstyr}
\label{Design_G_Development_EquipmentList}
\end{figure}

\begin{figure}[h!]
  \centering
    \includegraphics[width=0.5\textwidth]{Appendix/GUI-Prototype/PaperMockup/UdstyrsListe}
  \caption{Skærmbilledet til tilføjelse og visning af udstyr}
\label{Design_G_Development_EquipmentList_Final}
\end{figure}

\subsubsection{Udstyrsliste}
Figuerne \ref{Design_G_Development_EquipmentList} og \ref{Design_G_Development_EquipmentList_Final} viser papermockupen og det endelige skærmbillede af listen over udstyr og tilføjelse af nyt udstyr til systemet. Til designet af skærmbilledet har vi brugt "Law of proximity" så det er tydeligt at de to funktioner i skærmbilledet ikke hører sammen. Den eneste ændringer der er blevet lavet til det sidste skærmbillede er at der er blevet tilføjet en checkbox, hvor man kan vælge om udstyret skal kunne udlånes, hvis den ikke tjekkes vil udstyret blive kvalificeret som inventar og det vil derfor ikke blive vist under muligt udstyrs der kan tilføjes til en booking. 