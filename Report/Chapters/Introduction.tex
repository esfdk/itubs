\chapter{Indledning}
\label{Intro}
Det har længe været på tale at få lavet et integreret system til IT-Universitetets booking af lokaler for interne og eksterne brugere. Den nuværende metode er besværlig og uoverskuelig, da planlægningen og håndteringen af bookinger foregår i flere forskellige systemer med Facility Management (FM) som centralt bindepunkt\cite[Kap. A]{kravspec}.
\\Det integrerede systems primære formål er at lette arbejdsbyrden for FM samt minimere fejl. 

Denne rapport er udarbejdet i forbindelse med et projekt løbende fra 15. februar til 22. maj, 2013. Målet med projektet var at udvikle en løsning til IT-Universitetets ønske om et bookingsystem. 

Projektet resulterede i en første udgave, som opfylder nogle af kravene fra en kravspefikation (beskrevet i \ref{Intro_kravspec}). Den første udgave af løsningen består af det grafiske design og implementationen. 
\\Vi evaluerer løsningen ud fra antal problemer i usability tests samt hvor stor en del af kravspecifikationen, løsningen opfylder. Desuden har vi lavet en kort undersøgelse af, hvilke alternativer IT-Universitetet har i forbindelse med anskaffelsen af et booking-system.
\\Vores konklusion består af en evalueringen af projektet samt en beskrivelse af de erfaringer, vi har gjort igennem processen.

\section{Kravspecifikationen}
\label{Intro_kravspec}
Systemet er baseret på en kravspecifikation udarbejdet på kurset "Anskaffelse og kravspecifikation" i efteråret 2012. Kravspecifikationen er udarbejdet af Stig Larsen (stla@itu.dk), Miki Ipsen (mikc@itu.dk), Garwun Jeffrey Lai (gjel@itu.dk) og Merete Larsen (mnol@itu.dk)\footnote{Vi anskaffede kravspecifikationen hos Jeffrey Lai.}. 

\section{Versionsstyring, database og service info}
\label{Intro_vs}
Vi har anvendt Git (på github.com) til versionsstyring. Derudover har vi brugt en virtuel server opsat af ITU til at hoste vores service og database.
\begin{my_description}
\item[GitHub repository] https://github.com/esfdk/itubs
\item[Service URL] http://rentit.itu.dk/RentIt12/Services/Service.svc
\item[Database] RentIt12 på http://rentit.itu.dk (brugernavn: RentIt12Db -- adgangskode: Zaq12wsx)
\item[Test Administrator] Brugernavn="Admin@BookIt.dk" Adganskode="zAq12wSx"
\item[Test Bruger] Brugernavn="test@bookit.dk" Adgangskode="qwerty123"
\end{my_description}