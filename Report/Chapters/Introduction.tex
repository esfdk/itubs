\chapter{Indledning}
\label{Intro}
Det har længe været på tale at få lavet et integreret system til IT-Universitetets booking af lokaler for interne og eksterne brugere. Den nuværende metode er besværlig og uoverskuelig, da planlægningen og håndteringen af bookinger foregår i flere forskellige systemer med Facility Management (FM) som centralt bindepunkt \cite[Kap. A]{kravspec}.
\\Det integrerede systems primære formål er at lette arbejdsbyrden for FM samt minimere fejl. 

Denne rapport er resultatet af et projekt, som havde til formål at udvikle en løsning til IT-Universitetets ønske om et bookingsystem.

Projektet resulterede i en første funktionel udgave, som besår af en WCF service ASP.net klient. Denne første udgave understøtter booking af lokaler, udstyr og forplejning, men mangler nogle af de funktioner, som IT-Universitetet søger, fx fakturering og statistik. Systemet har en del test, som fejler, da systemet ikke er fuldt implementeret.
\\Den første udgave af bookingsystemet skal bruges af interne brugere (studerende og ansatte) på IT-Universitetet. Listen over forslag til fremtidige udgaver består bl.a. af en forbedret teststrategi, understøttelse af flere arbejdsopaver samt usability test.

En overfladisk undersøgelse af alternative systemer tyder på, at der ikke er et system på markedet, som opfylder alle IT-Universitetets krav til et bookingsystem.