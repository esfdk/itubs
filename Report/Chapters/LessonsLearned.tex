\chapter{Lessons learned}
\label{Conclusion_Lessons}
\section{Kravspecifikationen}
\label{Lessons_Krav}
I begyndelsen af projektet så vi kravspecifikationen som en række af fejlfri krav, som ikke kunne diskuteres eller bøjes. Vi forsøgte at følge kravspecifikationen i dets fulde omfang i stedet for at opfylde de krav, som vi mente var de vigtigste.

Da vi havde nærlæst kravspecifikationen, fandt vi ud af, at der var nogle problemer i den. For eksempel har vi lavet nogle rettelser i datamodellen, som var nødvendige i forbindelse med at lave en god implementering af systemet.

Lektien for os har været, at udvikleren skal være opmærksom på, hvad kravspecifikationen indeholder, og være klar til at diskutere kravene med kunden, så man er sikker på, at det bedste produkt bliver lavet. Desuden bør man diskutere en prioriteringsliste med kunden, således at kunden er sikker på, at udvikleren laver det vigtigste først.

\section{Projekter af samme natur er ikke nødvendigvis ens}
\label{Lessons_Projekt}
Da vi begge havde været med til at udvikle et system til filmudlejning, endte vi med at undervurdere, hvor meget arbejde der lå i implementeringen af vores design af bookingsystemet.

Systemet var udviklet i C\# med en WCF-service hooket op til en Microsoft SQL-database med en tilhørende klient. Da dette system delte mange af de samme elementer som systemet, vi har udviklet i forbindelse med dette projekt, vurderede vi, at det ikke ville være en særligt stor omkostning at overføre meget af funktionaliteten til vores bookingsystem.

Vi glemte at tage med i vores vurdering, at vi var fem personer på det forrige projekt, og derfor havde vi specialiseret os i forskellige dele af systemet. Da vi kun var to personer til dette projekt, betød det, at vi manglede erfaringen fra de personer fra det forrige projekt, som havde specialiseret sig inden for de ting, som vi ikke havde haft fokus på. Dette betød, at vi selv blev nødt til at tilegne os viden fra de felter, som de havde stået for. Dette gjorde, at vores vurdering af omkostningerne for vores valgte løsning var forkert.

Lektien i denne forbindelse har været, at man skal passe på med at fejlvurdere, hvor meget erfaring fra tidligere projekter kan gavne en i et andet projekt, selvom projekterne er meget ens.

\section{En service er ikke altid den rigtige løsning}
\label{Lessons_Service}
I forbindelse med vores valg af implementeringsstrategi, lavede vi en vurdering af, hvor meget det ville gavne kunden (ITU), hvis vi udviklede en service som en del af systemet. Vi vurderede, at det ville være en stor fordel for kunden, hvis det var nemt at udvide systemet til flere platforme.

Den første udgave, som vi udviklede i forbindelse med dette projekt, havde dog slet ikke brug for en service. Vi burde i stedet have koblet vores klient direkte til databasen. 
\\Arkitekturen, vi anvender i klienten, gør det relativt smertefrit at koble klienten op til en service i stedet for direkte til databasen, hvis systemet senere skulle udvides.

Vi tog ikke med i vores overvejelser, at projektperioden var begrænset, så vi vurderede kun omkostningerne i vores cost/benefit-analyse i forhold til hinanden. Dette betød, at vi ikke havde lavet en egentlig vurdering af, om det var muligt for os at nå at udvikle både service og klient.

I fremtidige projekter vil vi sørge for at vurdere, om der overhovedet er brug for avancerede features (fx en service) i forhold til kundes behov.

\section{Projektstyring}
\label{Lessons_Styring}
Vores styring af projektet har været meget løs og parallel. 

Det havde været en fordel for vores projekt, hvis vi havde haft delmilestones eller en iterativ styringsmodel. Vi ville have haft bedre styr på, hvor meget vi manglede på et givent tidspunkt i processen. Desuden ville vi haft mere materiale at få feedback på fra vores vejleder.

Da vi har arbejdet meget parallelt, har vi ikke været gode nok til at holde hinanden opdateret på, hvor man var i forbindelse med det, man arbejdede på. Når vi endelig tog beslutninger, blev de ikke dokumenteret ordenligt.

Vi manglede et fælles beslutningspapir, som var til rådighed for os begge, hvor vi skrev alle vores beslutninger ned i forhold til tekniske og grafiske designbeslutninger. Dette ville også have hjulpet os i forhold til vores meget parallelle arbejdsgang, da man ville have mulighed for at konsultere beslutningspapiret i stedet for at forstyrre den, man arbejder sammen med.

Lektien, vi har lært i denne forbindelse, har været, at selvom man arbejder parallelt, er det stadig vigtigt at have fælles mål. Derudover er det vigtigt at sørge for, at der stadig bliver kommunikeret mellem gruppemedlemmerne og sørge for, at den kommunikation bliver dokumenteret, så man kan støtte sig op af den senere i processen.