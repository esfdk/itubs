\section{Sammenligning med alternative systemer}
Der findes allerede systemer til booking af lokaler. I dette afsnit vil vi kigge kort på tre af disse og sammenligne dem med vores egen løsning.

\susbsection{Room booking system}
"Room booking system" er et webbaseret system som er designet til både at kunne opfylde skolers og virksomheders behov i forhold til booking af lokaler. Interfacet er designet rundt om en kalender, der viser de allerede eksisterende bookinger i farvekoder.Systemet understøtter ikke booking af udstyr eller forplejning og det er ikke koblet op til et AD og der vil derfor være brug for yderligere systemer for at kunne opfylde alle ITU's behov. Et abonnement som understøtter 200 rum og 500 brugere koster 434 kr. om måneden. Ulempen ved at vælge "Room booking system" i forhold til vores løsning er som nævnt, at det hverken understøtter forplejning eller udstyrs booking, der skal derfor ligges en hel del ressourcer i at få inkorporeret det i løsningen, hvis det da overhovedet er muligt.
Hjemmeside til Room booking system: http://www.roombookingsystem.co.uk/

\subsection{School booking}
"School booking" er et webbaseret sysemt som primært er designet til brug af skoler. Systemet giver mulighed for både at booke lokaler og udstyr, derudover er der mulighed for, at man kan tilføje ressourcer som eksempelvis kan være forplejning. Systemet giver også mulighed for statistisk udtræk over, hvor ofte en ressource bliver booket samt af hvem, men det understøtter ikke adgang til et AD. GUI'en er simpel og fungerer ligesom vores system, hvor man i et gitter vælger den ressource man gerne vil booke. Den årlige pris for at have "School booking" kørende for en instution på ITU's størrelse er omkring 11.000 kr. hvilket inkluderer mulighed for opsætning af 1000 ressourcer, 1000 brugere, samt mulighed for statestik og tilladelse til, at eksterne kan bruge systemet. "School booking" er klart den dyreste løsning af dem vi nævner, men den er samtidig den bedste. Med undtagelse af opkobling til AD og en ordentlig integration til kantinen understøtter det alle de funktioner som ITU kræver af et booking system. Sammenlignet med vores system, skal der ikke særligt meget arbejde til for, at "School booking" vil være en ligeså god løsning, det kan dog blive problematisk at koble AD op til det, da det skal være hostet på ITU's server for at fungere. 
Hjemmeside til School Booking: http://www.schoolbooking.com/

\subsection{Outlook}
Outlook giver mulighed for at resevere lokaler, når man inviterer folk til møder, denne løsning giver dog kun mulighed for at booke lokaler og giver ikke et særligt godt overblik, da de enkelte lokaler skal tilvælges og man derfor ikke får et fuldt overblik. Løsningen ville dog være nem at implementere da ITU er gået over til at bruge microsoft 365, som understøtter outlook som mail klient. Den store ulempe ved at vælge Outlook i forhold til vores system er, at det ikke nemt og automatisk giver brugeren et fuldt overblik over, hvilke lokaler der er ledige og hvilke der er booket. Derudover understøtter Outlook hverken forplejning eller udstyrsbooking og der skal derfor ligges meget arbejde i at udvikle systemer ved siden af til at understøtte de funktioner.



