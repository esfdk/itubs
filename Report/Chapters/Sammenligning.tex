\chapter{Sammenligning med alternative systmer}
Der findes allerede systemer til booking af lokaler, i dette afsnit ville vi kigge kort på tre af dem og sammenligne dem med vores egen løsning.
\secion{Room booking system}
Er et webbaseret system som er designet til både at kunne opfylde skoler og virksomheders behov i forhold til booking af lokaler. Interfacet er designet rundt om en kalender der viser de allerede eksisterende bookinger i farvekoder, systemet understøtter ikke booking af udstyr eller forplejning og der vil derfor være brug for yderlige systemer for at kunne opfylde alle ITU's behov. Et abonnement som understøtter 200 rum og 500 bruger koster 434 kr. om måneden
Hjemmeside til Room booking system: http://www.roombookingsystem.co.uk/

\section{Online school booking}
Er et webbaseret sysemt som primært er designet til brug af skoler, systemet giver mulighed for både at booke lokaler og udstyr men understøtter ikke bestilling af forplejning. Systemet er tilgengæld billigt det koster kun 750 kr. om året det vides dog ikke hvor meget opsætningen koster. 
Hjemmeside til Online School Booking: http://www.onlineschoolbooking.com/

\section{Outlook}
Outlook giver mulighed for at resevere lokale når man invitere folk til møder, denne løsning giver kun mulighed for at booke lokaler og den giver heller ikke et særligt godt overblik over hvad der er booket i forvejen. Løsningen ville dog være nem at implementere da ITU er gået over til at bruge microsoft 365, som understøtter outlook som mail client.