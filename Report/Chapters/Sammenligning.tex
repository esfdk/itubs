\section{Sammenligning med alternative systmer}
Der findes allerede systemer til booking af lokaler, i dette afsnit ville vi kigge kort på tre af dem og sammenligne dem med vores egen løsning.

\susbsection{Room booking system}
"Room booking system" er et webbaseret system som er designet til både at kunne opfylde skoler og virksomheders behov i forhold til booking af lokaler. Interfacet er designet rundt om en kalender der viser de allerede eksisterende bookinger i farvekoder, systemet understøtter ikke booking af udstyr eller forplejning og det er ikke koblet op til et AD og der vil derfor være brug for yderlige systemer for at kunne opfylde alle ITU's behov. Et abonnement som understøtter 200 rum og 500 bruger koster 434 kr. om måneden. Uplempen ved at vælge "Room booking system" i forhold til vores løsning er som nævnt at de hverken understøtter forplejning eller udstyr booking, der skal derfor ligges en hel del resourser i at få inkorporet det i løsningen hvis det da overhovedet er muligt.
Hjemmeside til Room booking system: http://www.roombookingsystem.co.uk/

\subsection{School booking}
"Online school booking" er et webbaseret sysemt som primært er designet til brug af skoler, systemet giver mulighed for både at booke lokaler og udstyr derudover er der mulighed for, at man kan tilføje ressourcer som også kan være forplejning. Systemet understøtter også statestik over hvor ofte en ressource bliver booket og af hvem, det understøtter dog ikke adgang ti et AD. GUI'en er simpel og fungrer ligesom vores system med at man vælge en ressource i et gitter som man gerne vil booke. Den årlige pris for at have School booking kørende for en instution på ITU's størrelse er omkring 11000 kr hvilket så inkluderer mulighed for opsætning af 1000 ressourcer, 1000 bruger, mulighed for statestik og tillades til at eksterne kan bruge systemet. School booking er klart den dyreste løsning af dem vi nævner men den er klart også den bedste, med undtagelse af opkobling til AD og en ordentlig integration til kantinen understøtter det alle de funktioner som ITU skal bruge af et booking system. Sammenlignet med vores system, skal der ikke særlig meget arbejde til for at "School booking" vil være en ligeså god løsning, det kan dog blive problematisk at koble AD optil det da det skal være hostet på ITU's server for at fungerer. 
Hjemmeside til School Booking: http://www.schoolbooking.com/

\subsection{Outlook}
Outlook giver mulighed for at resevere lokale når man invitere folk til møder, denne løsning giver kun mulighed for at booke lokaler og den giver heller ikke et særligt godt overblik over hvad der er booket i forvejen. Løsningen ville dog være nem at implementere da ITU er gået over til at bruge microsoft 365, som understøtter outlook som mail client. Den store ulempe ved at vælge outlook i forhold til vores system er, at det ikke giver brugen et overblik over hvilke lokaler der er ledige og hvilke der booket, derudover understøtter det hverken forplejning eller udstyrsbooking og der skal derfor ligges utroligt meget arbejde i at udvikle systemer ved siden af til at understøtte de funktioner.