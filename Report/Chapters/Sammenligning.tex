\section{Sammenligning med alternative systmer}
Der findes allerede systemer til booking af lokaler, i dette afsnit ville vi kigge kort på tre af dem og sammenligne dem med vores egen løsning.
\susbsection{Room booking system}
"Room booking system" er et webbaseret system som er designet til både at kunne opfylde skoler og virksomheders behov i forhold til booking af lokaler. Interfacet er designet rundt om en kalender der viser de allerede eksisterende bookinger i farvekoder, systemet understøtter ikke booking af udstyr eller forplejning og det er ikke koblet op til et AD og der vil derfor være brug for yderlige systemer for at kunne opfylde alle ITU's behov. Et abonnement som understøtter 200 rum og 500 bruger koster 434 kr. om måneden. Uplempen ved at vælge "Room booking system" i forhold til vores løsning er som nævnt at de hverken understøtter forplejning eller udstyr booking, der skal derfor ligges en hel del resourser i at få inkorporet det i løsningen hvis det da overhovedet er muligt.
Hjemmeside til Room booking system: http://www.roombookingsystem.co.uk/

\subsection{Online school booking}
"Online school booking" er et webbaseret sysemt som primært er designet til brug af skoler, systemet giver mulighed for både at booke lokaler og udstyr men understøtter ikke bestilling af forplejning eller adgang til AD. Gui'en er fortrinsvis simple og bruger ligesom Room booking system farvekoder. "Online school booking" er dog en del billigere end Room booking system da  det kun koster 750 kr. dette er dog uden opsætning. I forhold til vores løsning manlger online school booking at kunne understøtte forplejning før det opfylder de samme krav som vi gør.
Hjemmeside til Online School Booking: http://www.onlineschoolbooking.com/

\subsection{Outlook}
Outlook giver mulighed for at resevere lokale når man invitere folk til møder, denne løsning giver kun mulighed for at booke lokaler og den giver heller ikke et særligt godt overblik over hvad der er booket i forvejen. Løsningen ville dog være nem at implementere da ITU er gået over til at bruge microsoft 365, som understøtter outlook som mail client. Den store ulempe ved at vælge outlook i forhold til vores system er, at det ikke giver brugen et overblik over hvilke lokaler der er ledige og hvilke der booket, derudover understøtter det hverken forplejning eller udstyrsbooking og der skal derfor ligges utroligt meget arbejde i at udvikle systemer ved siden af til at understøtte de funktioner.